\chapter{Ringraziamenti}
\label{ch:Ringraziamenti}

Vorrei a questo punto spendere qualche parola per ringraziare tutte le persone che direttamente o indirettamente sono state parte di questo viaggio ed hanno contribuito ai risultati che ho raggiunto.
Inizio dal ringraziare l'Istituto Italiano di Tecnologia ed in particolare Daniele Pucci per avermi accolto in questa grande famiglia, dandomi la possibilità di vivere un'esperienza bellissima.
Un grazie anche a tutto il dipartimento di Artificial Mechanical Intelligence, in particolare a Silvio, Stefano e Giorgio che con la massima pazienza e professionalità mi hanno insegnato ad affrontare e risolvere i problemi (anche quelli impossibili).
Grazie anche ovviamente a tutti gli altri ragazzi di AMI, perché nemmeno nella più rosea delle aspettative avrei immaginato di trovare degli amici oltre che colleghi.

Ringrazio il Politecnico di Milano ed il Prof. Francesco Braghin, sempre disponibile e presente durante tutto il mio lavoro.

Approfitto di questa occasione per ringraziare la mia famiglia iniziando dai miei genitori Sergio e Maria Antonietta, consapevole che non basterebbe un'altra tesi di laurea per descrivere quanto sono stato fortunato.
Da quando sono piccolo hanno sempre camminato dietro di me, lasciando che fossi io a decidere la strada, non troppo vicini per farmi sentire libero, ma nemmeno troppo lontani per essere pronti ad aiutarmi se cadevo: questo percorso mi ha fatto crescere e sono fiero dell'uomo che sono diventato perché somiglio a voi.

Insieme ai miei genitori ringrazio mio fratello Enrico, nonostante i caratteri diversi, abituarmi a vivere in una camera da solo è stata una delle sfide più difficili da affrontare, ma so che nonostante la distanza ci saremo sempre l'uno per l'altro.

Grazie ai miei nonni Angelo, Emilia, Iolanda e Tonino: i professori mi hanno insegnato a risolvere le equazioni, ma voi mi avete insegnato a leggere e contare, anche se non vi chiamo spesso, siete sempre con me.

Ringrazio i miei zii Fausto e Irene e mio cugino Augusto, siete sempre stati la mia seconda famiglia, il rifugio quando facevo arrabbiare mamma e papà, perché la regola che valeva era: "quando ci sono gli zii Davide può fare tutto".

Un ultimo ringraziamento infine a tutti i miei amici: da quelli che mi conoscono da sempre e sono sempre stati presenti durante tutto il mio percorso, a quelli che sono entrati e poi usciti perché la vita è strana, per finire con quelli che ci sono da poco, ma non per questo meno importanti.
Tutti mi avete lasciato qualcosa di vostro e spero anche io di aver lasciato qualcosa di mio a voi, con alcuni ho condiviso i momenti più belli e più brutti della mia vita, sappiate che non me ne dimenticherò mai.



\begin{flushright}
Con affetto,
Davide
\end{flushright}